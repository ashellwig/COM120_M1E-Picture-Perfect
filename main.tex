\documentclass[stu,12pt]{apa7}
  \usepackage{times}               % Times New Roman Font Face
  \usepackage[american]{babel}     % Localization
  \usepackage[utf8]{inputenc}      % Input Encoding
  \usepackage{hyperref}            % Hyperlinks
  \usepackage{enumitem}            % Additional Enumeration Environment Settings
  \usepackage{geometry}            % Page Layout
  \usepackage{soul}                % Text Highlighting
  \usepackage{graphicx}            % Images
  \usepackage{csquotes}            % Quoting Environment
  \usepackage{bookmark}            % Required by `csquotes'
  \usepackage{mdframed}            % Colorful Tex-Box Environment
  \usepackage[toc]{appendix}       % Appendix
  \usepackage{fancyhdr}            % Headings and Footers
  \usepackage[%
    style=apa,%
    sortcites=true,%
    sorting=nyt%
  ]{biblatex}
  \usepackage{xcolor}

  % Bibliography Setup
  %% Language Mappings
  \DeclareLanguageMapping{english}{english-apa}
  \DeclareLanguageMapping{american}{american-apa}
  %% Bibliography File Path
  \addbibresource{main.bib}
  %% Categories for Specified Bibliography Items
  %%% Category for sources not referenced in-text
  \DeclareBibliographyCategory{consulted}
  \addtocategory{consulted}{noauthor_primer_2012}
  \addtocategory{consulted}{noauthor_communication_2013}

  % Hyperlink Setup
  \hypersetup{
    colorlinks = true,
    urlcolor = blue,
    linkcolor = blue,
    citecolor = blue
  }

  % Page and Text Layout
  \geometry{%
    a4paper,%
    top=1in,%
    bottom=1in,%
    left=1in,%
    right=1in%
  }
  \setlength{\headheight}{15pt}

  % Header
  \lhead{COM120CG1-M1D1}

  % Title Page
  \title{%
    Picture Perfect
  }
  \shorttitle{Module 1 Essay Assignment}
  \author{Ashton Hellwig}
  \authorsaffiliations{Department of Mathematics, Front Range Community College}
  \course{COM115: Interpersonal Communication}
  \professor{Richard Thomas}
  \duedate{November 08, 2020 23:59:59 MDT}
  \date{\today}
  \abstract{%
    \textbf{Overview}\\%
    Perception and self-concept are two of the most difficult parts of
      interpersonal communication. How we view ourselves and how we think others
      see us play central roles in how we communicate. If we assume that people
      are concentrating on our flaws rather than on our message, we may choose
      not to communicate at all. In this assignment, you are going to think
      critically about how perception and self-concept play a role in our
      personal and professional lives. You should spend approximately 6.5 hours
      on this assignment.
  }


\begin{document}
  % Title Page
  \maketitle

  \section{Defining Self-Concept and Perception}
    Self-Concept and perception go hand-in-hand. In simplest terms,
      \hl{self-concept} is how one perceives themselves as a person whereas
      \hl{perception} is how others perceive another individual as a person.
      These terms are synonymous concepts such as ``self image''. Both of these
      terms are important when discussing interpersonal communication due to
      how they affect many other factors for how well an idea will be
      communicated to the other party, which we will discuss later.
      I say that Self-Concept and perception go hand-in-hand because, more
      often than not, those with a higher level of positive self-concept will
      also perceive others in a positive light as well
      \parencite{markus_role_1985}


  \section{The Interaction Between King George and President Roosevelt}
    The interaction we witnessed between that of King George and President
      Roosevelt was an excellent illustration of solid interpersonal
      communication skills. King George's self-concept was initially in a
      seemingly poor state. He believed that he was seen as a weak leader
      to his subjects, or that he was not \textit{fit} to be leader to his
      citizen because of his stutter. This heavily tarnishes King George's
      self-image. We can even see that while he is talking and begins to
      stutter, he ends up getting even more frustrated which in turn makes the
      stuttering worse \parencite{michell_hyde_2012}. President Roosevelt
      attempts to change King George's self-concept by confidently speaking to
      him whilst moving about the room they are in without the aid of his
      wheelchair \parencite{michell_hyde_2012}. In reality, I am not sure if
      this act would really be seen in the way that Roosevelt intended. In many
      situations in which someone is experiencing a hardship and describing the
      situation to someone else, they would not necessarily be overjoyed by the
      act of someone ``showing-off'' their own pain and how well they can handle
      it compared to someone else. This could be seen as pontificating and
      pandering, which was described to us as something to avoid when attempting
      to be effective in interpersonal communication and ensuring that the
      other individual's psychological context remains in a safe space
      \parencite{headlee_10_2015}.


  \section{The Importance of Self-Concept and Perception in Communication}
    Self-concept and perception are huge components to effective and efficient
      interpersonal communication. Everyone has had their bad days and we all
      know that during those bad days it is difficult to feel ``good'' about
      anyone else, be it in their successes or failures. When we feel good
      about ourselves, it is far easier for us to feel good for others
      \parencite{pan_exploring_2019}. In ``\textit{Hyde Park on Hudson}'',
      President Roosevelt's perception of others was that the citizens and
      subjects either man is in charge of are not there to find mistakes and
      flaws, but rather guidance and leadership. I am not entirely sure if that
      is actually the case. Nowadays, anyone someone turns on the news one would
      be hard-pressed to find a story which is not describing the mishaps and
      mistakes of our country's politicians and business leaders. It seems that
      people specifically target flaws before all else when determining how to
      feel about any situation. There are many reasons we as humans do this, but
      I believe the main reason is that when other people make mistakes it
      allows us to feel better about the mistakes we have made ourselves
      \parencite{gerd_antos_handbook_2008}.


  % Bibliography
  %% Works Cited
  \newpage
  \printbibliography[%
    title={Works Cited},%
    heading={bibintoc},%
    notcategory={consulted}%
  ]

  %% Works Consulted
  \newpage
  \nocite{*}
  \printbibliography[%
    title={Works Consulted},%
    heading={bibintoc},%
    category={consulted}%
  ]
\end{document}
